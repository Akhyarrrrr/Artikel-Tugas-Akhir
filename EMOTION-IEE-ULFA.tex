\documentclass[conference,a4paper]{IEEEtran}
\IEEEoverridecommandlockouts
\usepackage[left=1.57cm,right=1.57cm,top=0.95cm,bottom=2.54cm]{geometry}
\newpage % Mulai halaman kedua
% \usepackage{caption}

% \captionsetup[figure]{justification=raggedright, singlelinecheck=off}

% Mengubah margin pada halaman kedua
\newgeometry{left=1.57cm,right=1.57cm,top=1.9cm,bottom=2.54cm}
% The preceding line is only needed to identify funding in the first footnote. If that is unneeded, please comment it out.
\usepackage{cite}
\usepackage{amsmath,amssymb,amsfonts}
\usepackage{algorithmic}
\usepackage{graphicx}
\usepackage{textcomp}
\usepackage{xcolor}
\usepackage{balance}
\usepackage{multirow}
\def\BibTeX{{\rm B\kern-.05em{\sc i\kern-.025em b}\kern-.08em
    T\kern-.1667em\lower.7ex\hbox{E}\kern-.125emX}}
\begin{document}

\title{
  Development of RWikiStat 4.0 for Modern Statistical Learning
  \\
  \thanks{*Corresponding Author}
}

\makeatletter
\newcommand{\linebreakand}{
  \end{@IEEEauthorhalign}
  \hfill\mbox{}\par
  \mbox{}\hfill\begin{@IEEEauthorhalign}
}
\makeatother

\author{
  \IEEEauthorblockN{Hizir Sofyan}
  \IEEEauthorblockA{\textit{Statistics Department} \\
    \textit{Universitas Syiah Kuala}\\
    Banda Aceh, Indonesia 23111\\
    hizir@usk.ac.id}
  \and
  \IEEEauthorblockN{Munawar}
  \IEEEauthorblockA{\textit{Statistics Department} \\
    \textit{Universitas Syiah Kuala}\\
    Banda Aceh, Indonesia 23111 \\
    munawar@mhs.usk.ac.id}
  \and
  \IEEEauthorblockN{Muhammad Subianto}
  \IEEEauthorblockA{\textit{Informatics Department} \\
    \textit{Universitas Syiah Kuala}\\
    Banda Aceh, Indonesia 23111\\
    subianto@usk.ac.id}
  \and
  \IEEEauthorblockN{Kurnia Saputra\textsuperscript{*}}
  \IEEEauthorblockA{\textit{Informatics Department} \\
    \textit{Universitas Syiah Kuala}\\
    Banda Aceh, Indonesia 23111\\
    kurnia.saputra@usk.ac.id}
  \and
  \linebreakand
  \IEEEauthorblockN{Naufal Mas Adha}
  \IEEEauthorblockA{\textit{Informatics Department} \\
    \textit{Universitas Syiah Kuala}\\
    Banda Aceh, Indonesia 23111\\
    naufal@mhs.usk.ac.id}
  \and
  \IEEEauthorblockN{Affan Ian Amara}
  \IEEEauthorblockA{\textit{Informatics Department} \\
    \textit{Universitas Syiah Kuala}\\
    Banda Aceh, Indonesia 23111\\
    affan@mhs.usk.ac.id}
  \and
  \IEEEauthorblockN{Muhammad Nurifai}
  \IEEEauthorblockA{\textit{Informatics Department} \\
    \textit{Universitas Syiah Kuala}\\
    Banda Aceh, Indonesia 23111\\
    nurifai@mhs.usk.ac.id}
  \and
  \IEEEauthorblockN{AkhULFA}
  \IEEEauthorblockA{\textit{Informatics Department} \\
    \textit{Universitas Syiah Kuala}\\
    Banda Aceh, Indonesia 23111\\
    akhULFA1@mhs.usk.ac.id}
}

\maketitle

\begin{abstract}
  RWikiStat 4.0 is a cross-platform application designed to modernize statistical
  learning by integrating cutting-edge technologies such as artificial
  intelligence, cloud computing, and offline support. The application supports
  web, Android, and iOS platforms, enabling seamless accessibility for users
  across devices. This paper details the architecture, features, and
  implementation of RWikiStat 4.0, highlighting its advancements over previous
  versions. Testing results demonstrate significant improvements in usability,
  performance, and security, making it a robust tool for modern statistical
  education.
\end{abstract}

\begin{IEEEkeywords}
  RWikiStat, Statistical Learning, Mobile Applications, Web-Based Education
\end{IEEEkeywords}

\section{Introduction}
\label{sect:introduction}

Rwikistat has been a pioneer in digital-based statistical learning since its
inception. The first version in 2010 utilized a combination of \textit{Rweb}
and \textit{MediaWiki} to provide an interactive statistics learning platform
accessible via intranet and internet \cite{b1}. Rwikistat 2.0, introduced in
2012, enhanced this approach by offering a \textit{Live CD/DVD} solution,
enabling operations without additional installation, thereby increasing its
flexibility \cite{b2}. Most recently, Rwikistat 3.0, launched in 2018, brought
statistical learning into the \textit{mobile} domain with an Android-based
platform, catering to the needs of modern users who prioritize portability and
convenience \cite{b3}.

While each version of Rwikistat has made significant innovations, challenges
remain:
\begin{itemize}
  \item \textbf{Dependency on Rweb}: The discontinuation of Rweb has necessitated an urgent shift to a more stable alternative \cite{b2,b3}.
  \item \textbf{Limited user interface}: Users have reported that the command line in Rwikistat 3.0 still requires improvements for better intuitiveness \cite{b3}.
  \item \textbf{Focus on local environments}: Many features rely on local installations or specific devices, limiting widespread adoption in modern cloud-based scenarios \cite{b2}.
\end{itemize}

To address these challenges, \textbf{Rwikistat 4.0} has been designed as an
integrated statistical learning platform that is cloud-based and more
responsive to user needs.

\subsection*{Solutions in Rwikistat 4.0}
\begin{itemize}
  \item \textbf{Cloud-Based Architecture}: Rwikistat 4.0 eliminates dependency on Rweb by leveraging a \textit{Docker}-based R server, enabling direct execution of R syntax via REST APIs, supporting scalability for various devices.
  \item \textbf{Enhanced User Interface}: The interface has been redesigned using \textit{React Native}, providing a smoother and more responsive experience on both web and mobile applications. Drag-and-drop functionality for data visualization empowers users without programming experience to interact directly with statistical tools.
  \item \textbf{Cloud-Based Collaboration and Storage}: Integration with services like Google Drive and Dropbox enables users to save and share analysis results directly, facilitating modern collaborative learning needs.
  \item \textbf{AI-Powered Statistical Learning Modules}: Rwikistat 4.0 includes artificial intelligence modules to provide automatic recommendations for relevant statistical methods based on user input data, transforming it from a passive tool to an active learning companion.
  \item \textbf{Cross-Device Accessibility}: The platform is fully cross-platform, ensuring seamless access on Windows, macOS, Android, and iOS devices.
\end{itemize}

Through these innovations, \textbf{Rwikistat 4.0} not only continues its
tradition as an adaptive statistics learning tool but also sets a new standard
for interactivity and accessibility in technology-driven education. This
platform is poised to advance statistical education, addressing future
challenges with relevant and innovative solutions.

\section{System Design}
\label{sect:system_design}

The design of RWikiStat 4.0 focuses on flexibility and scalability to support
multiplatform usage. Its architecture consists of the following components:

\subsection{Frontend}
\begin{itemize}
  \item \textbf{Web Application:} Developed using Next.js, it ensures fast rendering and optimal performance.
  \item \textbf{Mobile Applications:} Built on React Native, the app provides a unified codebase for Android and iOS.
\end{itemize}

\subsection{Backend}
\begin{itemize}
  \item \textbf{API Layer:} Powered by Express.js, it facilitates robust RESTful services.
  \item \textbf{Database:} MongoDB serves as the primary database, offering flexibility and scalability for diverse datasets.
\end{itemize}

\subsection{Key Features}
\begin{itemize}
  \item An AI-powered chatbot for statistical guidance.
  \item Real-time R script compilation integrated with Shiny for interactive analytics.
  \item Discussion forums for collaborative learning.
  \item Offline access with Progressive Web App (PWA) technology.
\end{itemize}

\subsection{Cloud Integration}
AWS and Firebase ensure secure storage, authentication, and real-time
synchronization, enhancing user experience.

\section{Testing and Validation}
\label{sect:testing_validation}

\subsection{Functional Testing}
Blackbox testing validated each feature, including:
\begin{itemize}
  \item Google-based and NIM/password-based login functionalities.
  \item Real-time R script compilation and error handling.
  \item Accessibility of learning modules and forums.
\end{itemize}

\subsection{Usability Testing}
Using the UMUX method, feedback was collected from 10 respondents. The average
usability score was 85.4\%, classifying the application as "Very Usable."

\subsection{Performance Testing}
JMeter was employed to test API throughput and stress scenarios. The
application successfully handled datasets exceeding 1GB without significant
performance degradation.

\subsection{Security Testing}
OAuth2 authentication and MongoDB encryption were validated to ensure data
security.

\section{Results and Discussion}
\label{sect:results_discussion}

The testing phase highlighted the following outcomes:
\begin{itemize}
  \item High satisfaction rates due to intuitive design and efficient functionalities.
  \item Positive reception of AI-guided learning modules for simplifying complex
        concepts.
  \item Challenges in iOS development due to platform-specific requirements.
\end{itemize}

While user feedback was overwhelmingly positive, enhancements are needed for
advanced statistical explanations and seamless iOS integration.

\section{Conclusion}
\label{sect:conclusion}

RWikiStat 4.0 represents a significant advancement in statistical learning
tools, combining multiplatform support, AI integration, and offline
capabilities. Future development will focus on enhancing AI functionalities and
incorporating advanced statistical techniques.

\balance

\begin{thebibliography}{00}
  \bibitem{b1}Subianto, Muhammad, and Hizir Sofyan. "Interactive statistics learning with RWikiStat." 2010 International Conference on Networking and Information Technology. IEEE, 2010.

  \bibitem{b2}Sofyan, Hizir, Edi Muttaqin, and Muhammad Subianto. "RWikiStat 2.0: a Web Based Statistical Learning System (Session 1B (IASC-ARS))." Proceedings of the symposium of Japanese Society of Computational Statistics 26. Japanese Society of Computational Statistics, 2012.

  \bibitem{b3}Sofyan, Hizir, et al. "Analisis Kepuasan Pengguna Aplikasi RWikiStat 3.0." Journal of Data Analysis 2.2 (2020): 80-87.

\end{thebibliography}

\end{document}